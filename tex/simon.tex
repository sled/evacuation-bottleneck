
\section{Exit Selection}
In emergency evcuation, the selection of the exit route is one of the most
important decisions. We take this into account in our simulation by the
implementation of the paper "Exit Selection with Best Response Dynamics"
\cite{BestResponseDynamics}. The paper describes an algorithm about how people
choose an appropriate exit based on the game theoretic concept of best
response dynamics. In the model the agents are the player and the strategies
are the possible target exits.

We assume that agents will select the fastest evacuation route. Despite of the
time related factor we include two other factors: familiarity and visibility
of the exits. The estimated evacuation time of an agent is the sum of the
estimated moving time and the estimated queueing time. The estimated moving
time is estimated simply by dividing the distance to the exit by the velocity
of the agent. The estimated queuing time depends on the exit's capacity and on
the number of the other agents that are heading towards the exit and are
closer to it than the agent itself. The estimated queuing time binds the
decision of a single agent to the decision of other agents. In conclusion,
this means the fastest exit route for a specific agent may change during the
evacuation.

The familiarity and visibility factor constrain the set of possible exits.
These factors can be seen as binary flags and the number of possible
combinations form the preference groups. Every door will be divided into a
preference group. Agents will select an exit from the nonempty group that has
the best preference. The doors in other preference groups are not of any
interest.

\subsection{Mathematical Formulation of the Model}
The agents are refered with indices $i$ and $j$, where $i,j \in \mathcal{N} =
\{1,2,3,...,N\}$. Exits can be seen as strategies, exits are denoted by $e_k,
k \in \mathcal{K} = \{1,2,...,K\}$. Strategies are denoted by $s_i \in
\{e_1,...,e_K\} = S_i, i \in \mathcal{N}$ where $S_i$ is a strategy set.

The agent's strategies are concluded by \[s := (s_{1},...,s_{N}) \in S_{1}
\times \cdot\cdot\cdot \times S_{N} = S \]  The strategies of all other agents
but agent $i$ is defined by \[s_{-i} := (s_{i},...,s_{i-1},s_{i+1},...,s_{N})
\in S_{-i}\] The estimated moving time depends on the agent $i$'s position
$\mathbf{r}_i$ and the exit $e_k$'s position $\mathbf{b}_k$. The positions of
the agents are in the set $\mathbf{r} := (\mathbf{r}_1,...,\mathbf{r}_N)$. So
the distance between agent $i$ and the exit $e_k$ is \[d(e_{k};\mathbf{r}_i) =
||\mathbf{r}_i - \mathbf{b}_k||\] The estimated moving time is the division of
the distance $d(e_{k};\mathbf{r}_i)$ by agent $i$'s velocity $v^{0}_i$
\[\tau_i(e_k;\mathbf{r}_i) = \frac{1}{v^{0}_i}d(e_k;\mathbf{r}_i)\] The
estimated queueing time is defined by the sum of all agents but agent $i$
heading towards exit $e_k$ and are closer to exit $e_k$ divided by the exit
$e_k$'s capacity $\beta_k$.

The subset of all agents $j \ne i$ who are closer to $e_k$ than agent $i$ is
given by \[\Lambda_i(e_k, s_{-i};\mathbf{r}) = \{j \ne i | s_j = e_k,
d(e_k;\mathbf{r}_j) \le d(e_k;\mathbf{r}_i)\}\] The number of elements in the
subset $\Lambda_i(e_k, s_{-i};\mathbf{r})$ is denoted by
\[\lambda_i(e_k,s_{-1};\mathbf{r}) = |\Lambda_i(e_k, s_{-i};\mathbf{r})| \]
The exit $e_k$'s capacity $\beta_k$ is a scalar value telling us how many
agents can pass the exit $e_k$ at once.

So the estimated queueing time is
\[\frac{1}{\beta_k}\lambda_i(e_k,s_{-1};\mathbf{r}) = |\Lambda_i(e_k, s_{-i};\mathbf{r})|\]
The sum of the estimated moving time and estimated queueing time gives us the
estimated evacuation time for agent $i$ through the exit $e_k$
\[T_i(s_i, s_{-i};\mathbf{r}) = \frac{1}{\beta_k}\lambda_i(e_k,s_{-1};\mathbf{r}) + \tau_i(e_k;\mathbf{r}_i)\]
As a result of the game theoretic principle, the strategy of agent $i$ is the
best response to the other agents' strategies. This means every agent will
choose the exit which has the lowest evacuation time.
\[s_i = BR_i(s_{-i};\mathbf{r}) = \arg \underset{s^{\prime}_i \in S_i}{\min} T_i(s^{\prime}_i,s_{-i};\mathbf{r}) \]
As we have mentioned before the effects of familiarity and visibility of exits
can constrain the group of possible exits for agent $i$, these conditions are
taken into account by defining two binary flags
\[fam_i(e_k),\ vis(e_k;\mathbf{r_i}),\quad \forall\ i \in \mathcal{N}, k \in K\]
The binary flags give certain information about agent $i$:
\[fam_i(e_k) \ = \ 
\begin{cases} 
1 & \text{if exit\ } e_k \text{\ is familiar to agent\ } i \\
0 & \text{if exit\ } e_k \text{\ is not familiar to agent\ } i
\end{cases}\]
\[vis(e_k; \mathbf{r}_i) = \ 
\begin{cases} 
1 & \text{if exit\ } e_k \text{\ is visible to agent\ } i \\
0 & \text{if exit\ } e_k \text{\ is not visible to agent\ } i
\end{cases}\]
These factors are the criterias for dividing the exits in to groups with
preference numbers. There are four possible combinations which means there are
four groups of exits with preference numbers from one to four. The smaller the
preference number is, the more preferable the exit. The familiarity of an exit
has a bigger influence about how preferable an exit is. Studies have shown
that evacuees prefere familiar routes even if there is a shorter route
\cite{BestResponseDynamics}. The visibility flag is important for the
calculation of the estimated queueing time beacause an agent is only able to
estimate the queue in front of a door if he can see the door.

According to the previous definition the doors will be grouped as shown in the table below.
\begin{center}
\begin{tabular*}{\textwidth}{@{\extracolsep{\fill}} cccc}
	\hline
	Preference number & Exit group &  $vis(e_k; \mathbf{r}_i)$ & $fam_i(e_k)$\\ 
	\hline

	1 & $E_i(1)$ & 1 & 1\\ 

	2 & $E_i(2)$ & 0 & 1\\ 

	3 & $E_i(3)$ & 1 & 0\\ 

	4 & No Preference & 0 & 0\\
	\hline
\end{tabular*}
\end{center}

\textbf{Table 1} The preference groups in which the exits will be divdided
into. The smaller the preference number, the more preferable the exit. The
fourth preference group describes people in panic which are not familiar with
the exits and can not see any either. \cite{BestResponseDynamics}

Mathematically the selection of the door is defined as
\[s_i = BR_i(s_{-i};\mathbf{r}) = \arg \underset{s^{\prime}_i \in S_i}{\min} T_i(s^{\prime}_i,s_{-i};\mathbf{r})\]
\[s^{\prime}_i \in E_i(\overline{z})\]
The specific agent $i$ chooses an exit from the non-empty Group
$E_i(\overline{z})$ which has the best preference number $\overline{z}$ for
him. 

In addition to the paper we added an extra patience factor. The patience
factor is a simple comparison between the evacuation time of the preferable
new exit and the previously chosen exit. This is needed because it may happen
that an exit in a better preference group gets in sight. Despite the fact that
the exit is in a better preference group the evacuation time could take much
longer. So the agent will not redecide if the evacuation time of the new
preferable exit is greater than the evacuation time of the agent's previous
decision. This could be omitted if the number of exits is significant higher
than the number of possible preference groups.




